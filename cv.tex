\documentclass[11pt]{article}
\usepackage{amsmath}%
\usepackage{amsfonts}%
\usepackage{amssymb}%
\usepackage{amsthm}%

\hoffset=-.04\textwidth%
\textwidth=1.08\textwidth%
\voffset=-0.03\textheight
\textheight=1.06\textheight
%\voffset=0.5cm%
%\textheight=1.08\textheight%

\newcommand{\C}{\mathbf{C}}%
\newcommand{\F}{\mathbf{F}}%
\newcommand{\Q}{\mathbf{Q}}%
\newcommand{\Qbar}{\overline{\Q}}%
\newcommand{\Z}{\mathbf{Z}}%
\newcommand{\R}{\mathbf{R}}%
\newcommand{\T}{\mathbf{T}}%
\renewcommand{\H}{\mathrm{H}}%

\newcommand{\eps}{\varepsilon}%
\newcommand{\con}{\equiv}%
\newcommand{\isom}{\cong}%
\newcommand{\rhobar}{\overline{\rho}}
\newcommand{\tensor}{\otimes}


% ---- SHA ----
\DeclareFontEncoding{OT2}{}{} % to enable usage of cyrillic fonts
  \newcommand{\textcyr}[1]{%
    {\fontencoding{OT2}\fontfamily{wncyr}\fontseries{m}\fontshape{n}%
     \selectfont #1}}%
\newcommand{\Sha}{{\mbox{\textcyr{Sh}}}}%
\newcommand{\tor}{\mbox{\scriptsize\rm tor}}

\newcommand{\myname}{William Stein}
\newcommand{\phone}{{\sf (206) 419-0925}}
\newcommand{\email}{{\sf wstein@gmail.com}}
\newcommand{\www}{{\sf http://wstein.org}}
\newcommand{\address}{}
\usepackage{fancyhdr,ifthen}
\pagestyle{fancy}
\cfoot{\thepage}  % no footers (in pagestyle fancy)
% running left heading
\lhead{\bfseries\LARGE\em \noindent{}\hspace{-.2em}\myname{}
        \hfill \thisdocument\vspace{-.2ex}\\}
% running right heading
%\newcommand{\spc}{1.31em}
\newcommand{\spc}{1em}

%\rhead{\em {\small{\phone{}}} \hfill $\cdot$\hfill
% \email{} \hfill $\cdot$\hfill \www{}}

\rhead{\em {\small{\phone{}}} \hfill \email{} \hfill \www{}}

\setlength{\headheight}{5ex}
\newcommand{\mainhead}[1]{\begin{center}{\Large \bf #1}\end{center}}
\newcommand{\head}[1]{\vspace{1.5ex}\par\noindent{\large \bf #1}\par\noindent}
\newcommand{\subhead}[1]{\vspace{2ex}\par\noindent{\sl #1}\vspace{1ex}\par\noindent{}}
\newcommand{\ptitle}{\sl}

\newcommand{\hra}{\hookrightarrow}


%%%% Theoremstyles
\theoremstyle{plain}
\newtheorem{theorem}{Theorem}[section]
\newtheorem{proposition}[theorem]{Proposition}
\newtheorem{corollary}[theorem]{Corollary}
\newtheorem{claim}[theorem]{Claim}
\newtheorem{lemma}[theorem]{Lemma}
\newtheorem{conjecture}[theorem]{Conjecture}

\theoremstyle{definition}
\newtheorem{definition}[theorem]{Definition}
\newtheorem{algorithm}[theorem]{Algorithm}
\newtheorem{question}[theorem]{Question}
\newtheorem{problem}[theorem]{Problem}
\newtheorem{goal}[theorem]{Goal}

\theoremstyle{remark}
\newtheorem{remark}[theorem]{Remark}
\newtheorem{remarks}[theorem]{Remarks}
\newtheorem{example}[theorem]{Example}
\newtheorem{exercise}[theorem]{Exercise}

\DeclareMathOperator{\End}{End}%
\DeclareMathOperator{\Tr}{Tr}%
\DeclareMathOperator{\Res}{Res}%
\DeclareMathOperator{\res}{res}%
\DeclareMathOperator{\BSD}{BSD}%
\DeclareMathOperator{\Gal}{Gal}%
\DeclareMathOperator{\GL}{GL}%
\DeclareMathOperator{\Aut}{Aut}%
\DeclareMathOperator{\Reg}{Reg}%
\DeclareMathOperator{\Vis}{Vis}%
\DeclareMathOperator{\Ker}{Ker}%
\DeclareMathOperator{\Coker}{Coker}%
\DeclareMathOperator{\Sel}{Sel}%
\DeclareMathOperator{\ord}{ord}%
\DeclareMathOperator{\new}{new}%
\DeclareMathOperator{\an}{an}%


\newcommand{\thisdocument}{RRF: Curriculum Vitae}

\begin{document}
%\maketitle

\head{Professional Preparation}%
\begin{center}
\begin{tabular}{lll}
  % after \\: \hline or \cline{col1-col2} \cline{col3-col4} ...
\mbox{}\hspace{3.2ex}
&  Harvard University & NSF Postdoc, 2000--2004 \\
&  University of California at Berkeley & Mathematics, Ph.D. 2000 \\
&   Northern Arizona University\hspace{1.03in}\mbox{}& Mathematics, B.S. 1994 \\
\end{tabular}
\end{center}

\head{Appointments}
\begin{itemize}\setlength{\itemsep}{-0.8ex}
\item Professor of Mathematics (with tenure),
University of Washington, September 2010--present.
\item Associate Professor of Mathematics (with tenure),
University of Washington, September 2006--2010.
\item Associate Professor of Mathematics (with tenure),
UC San Diego, July 2005--June 2006.
\item Benjamin Peirce Assistant  Professor  of Mathematics, 
Harvard University, July 2001--May 2005.
\item NSF Postdoctoral Research Fellow 
under Barry Mazur at Harvard University, August 2000--May 2004.
\item Clay Mathematics Institute Liftoff Fellow, Summer 2000.
\end{itemize}

\head{Synergistic Activities}
\begin{itemize}\setlength{\itemsep}{-0.5ex}

\item {\bf Research Tools:} Founder and director of Sage
  (\url{http://sagemath.org}), which is a large free open source software
  project (well over 100,000 lines of the Sage code was written by the
  PI).  Author of the modular forms, modular symbols, and modular
  abelian varieties parts of the Magma computer algebra system (over
  25,000 lines of code and documentation).
%These
%  are tools used by mathematicians who do computations with modular
%  forms.%

%\item {\bf Databases:} Cre Modular Forms
%Database. This contains continually expanding data about
%elliptic curves and modular forms: 
%{\tt http://www.wstein.org/Tables/}.

\item {\bf Outreach:} SIMUW 2006, 2007, 2008, 2012  math
  camps; Canada/USA MathCamp mentor (2002); Math Circles talks in
  Boston; 2011 REU on elliptic curves; involved many
  undergraduates in work on the Sage software.
\end{itemize}

\head{Graduate and Postdoctoral Advisors}
\begin{itemize}\setlength{\itemsep}{-0.5ex}
\item {\bf NSF Postdoctoral advisor:} Barry Mazur, Harvard
University.
\item {\bf Ph.D. advisor:} Hendrik Lenstra, University of Leiden,
Netherlands.%
\end{itemize}
\head{Thesis Students:}
\begin{itemize}\setlength{\itemsep}{-0.5ex}
\item \textbf{Thesis Students}  4 Ph.D. students at UW:  
Robert Bradshaw's 2010 Ph.D. on {\em Provable Computation of Motivic
$L$-functions}; Robert Miller's 2010 Ph.D. on {\em Computational Verification of
the Birch and Swinnerton-Dyer Conjecture};
currently advising Alyson Dienes's Ph.D. thesis on
{\em Modular degrees of elliptic curves over totally real fields}
and Simon Spicer's Ph.D. thesis work.
Advised eight undergraduate senior theses at
Harvard and three at UW.
\end{itemize}

% \newpage

% \begin{center}
% \head{Biographical Statement}
% \end{center}

% \noindent{}William Stein will contribute to this project in both a
% research and managerial role.  Stein has been a driving force over the
% last 10 years in applications of computation to research on modular
% forms, $L$-functions, and associated arithmetic objects.  As director
% of the Sage project (http://sagemath.org), he has experience managing
% working groups and working with undergraduates on a wide range of
% projects.

% \vspace{2ex}


% \noindent{}{\bf Support Statement:}
% The proposed project naturally fits in with my other
% NSF-funded research on the Birch and Swinnerton-Dyer conjecture, from
% which I will receive 2 months summer support during the next 2 years
% and funding for travel and materials (DMS-0653968).  I have also
% received an NSF grant (DMS-0703583) to support one postdoc for three
% years, who will work on developing exact linear algebra algorithms and
% implementations for Sage.  I am a co-PI on the Arizona Winter School
% grant (DMS-0602287); this is a yearly 1-week 120-person graduate
% student workshop in arithmetic geometry, whose upcoming topics mesh
% well with the themes of the current proposal (e.g., the next theme is
% quadratic forms, and theta series of quadratic forms are modular
% forms).  I will also be directing graduate and undergraduate research
% that is related to this project during the academic year, not just
% during the summer.  Finally, I am also applying to the NSF FRG program
% for additional money to support a postdoc and workshops.

\end{document}
