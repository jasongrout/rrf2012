\documentclass[11pt]{article}
\usepackage{url}
\usepackage{graphicx}
\usepackage{amsmath}%
\usepackage{amsfonts}%
\usepackage{amssymb}%
\usepackage{amsthm}%

\hoffset=-.04\textwidth%
\textwidth=1.08\textwidth%
\voffset=-0.03\textheight
\textheight=1.06\textheight
%\voffset=0.5cm%
%\textheight=1.08\textheight%

\newcommand{\C}{\mathbf{C}}%
\newcommand{\F}{\mathbf{F}}%
\newcommand{\Q}{\mathbf{Q}}%
\newcommand{\Qbar}{\overline{\Q}}%
\newcommand{\Z}{\mathbf{Z}}%
\newcommand{\R}{\mathbf{R}}%
\newcommand{\T}{\mathbf{T}}%
\renewcommand{\H}{\mathrm{H}}%

\newcommand{\eps}{\varepsilon}%
\newcommand{\con}{\equiv}%
\newcommand{\isom}{\cong}%
\newcommand{\rhobar}{\overline{\rho}}
\newcommand{\tensor}{\otimes}


% ---- SHA ----
\DeclareFontEncoding{OT2}{}{} % to enable usage of cyrillic fonts
  \newcommand{\textcyr}[1]{%
    {\fontencoding{OT2}\fontfamily{wncyr}\fontseries{m}\fontshape{n}%
     \selectfont #1}}%
\newcommand{\Sha}{{\mbox{\textcyr{Sh}}}}%
\newcommand{\tor}{\mbox{\scriptsize\rm tor}}

\newcommand{\myname}{William Stein}
\newcommand{\phone}{{\sf (206) 419-0925}}
\newcommand{\email}{{\sf wstein@gmail.com}}
\newcommand{\www}{{\sf http://wstein.org}}
\newcommand{\address}{}
\usepackage{fancyhdr,ifthen}
\pagestyle{fancy}
\cfoot{\thepage}  % no footers (in pagestyle fancy)
% running left heading
\lhead{\bfseries\LARGE\em \noindent{}\hspace{-.2em}\myname{}
        \hfill \thisdocument\vspace{-.2ex}\\}
% running right heading
%\newcommand{\spc}{1.31em}
\newcommand{\spc}{1em}

%\rhead{\em {\small{\phone{}}} \hfill $\cdot$\hfill
% \email{} \hfill $\cdot$\hfill \www{}}

\rhead{\em {\small{\phone{}}} \hfill \email{} \hfill \www{}}

\setlength{\headheight}{5ex}
\newcommand{\mainhead}[1]{\begin{center}{\Large \bf #1}\end{center}}
\newcommand{\head}[1]{\vspace{1.5ex}\par\noindent{\large \bf #1}\par\noindent}
\newcommand{\subhead}[1]{\vspace{2ex}\par\noindent{\sl #1}\vspace{1ex}\par\noindent{}}
\newcommand{\ptitle}{\sl}

\newcommand{\hra}{\hookrightarrow}


%%%% Theoremstyles
\theoremstyle{plain}
\newtheorem{theorem}{Theorem}[section]
\newtheorem{proposition}[theorem]{Proposition}
\newtheorem{corollary}[theorem]{Corollary}
\newtheorem{claim}[theorem]{Claim}
\newtheorem{lemma}[theorem]{Lemma}
\newtheorem{conjecture}[theorem]{Conjecture}

\theoremstyle{definition}
\newtheorem{definition}[theorem]{Definition}
\newtheorem{algorithm}[theorem]{Algorithm}
\newtheorem{question}[theorem]{Question}
\newtheorem{problem}[theorem]{Problem}
\newtheorem{goal}[theorem]{Goal}

\theoremstyle{remark}
\newtheorem{remark}[theorem]{Remark}
\newtheorem{remarks}[theorem]{Remarks}
\newtheorem{example}[theorem]{Example}
\newtheorem{exercise}[theorem]{Exercise}

\DeclareMathOperator{\End}{End}%
\DeclareMathOperator{\Tr}{Tr}%
\DeclareMathOperator{\Res}{Res}%
\DeclareMathOperator{\res}{res}%
\DeclareMathOperator{\BSD}{BSD}%
\DeclareMathOperator{\Gal}{Gal}%
\DeclareMathOperator{\GL}{GL}%
\DeclareMathOperator{\Aut}{Aut}%
\DeclareMathOperator{\Reg}{Reg}%
\DeclareMathOperator{\Vis}{Vis}%
\DeclareMathOperator{\Ker}{Ker}%
\DeclareMathOperator{\Coker}{Coker}%
\DeclareMathOperator{\Sel}{Sel}%
\DeclareMathOperator{\ord}{ord}%
\DeclareMathOperator{\new}{new}%
\DeclareMathOperator{\an}{an}%


\newcommand{\thisdocument}{RRF: Proposed Research}

\begin{document}
\section{Introduction and Rationale}

%[[Provide brief critical review of the pertinent literature, theoretical
%background, and justification for the proposed research. Describe any
%results already achieved, including publications.]]

Most of the PI's published research work is in
number theory, which is an area of pure mathematics.  The PI embarked
on a new and risky research path at Harvard in 2005, when he started
the Sage mathematical software project, which has greatly grown since
he was recruited by UW in 2006.  There are now around 500 Sage
developers, over 10,000 downloads of Sage per month, and over 55,000
unique visitors each month to the \url{http://sagemath.org} website.
Sage is a free self contained program that you can download and run on
your computer, and the goal of the project is to create a viable free
open source alternative to the commercial systems Maple, Mathematica,
Matlab and Magma.

The Sage notebook, whose development was partly supported by the RRF
program in 2009, is the primary user interface to Sage.  It was
designed mainly for use by a relatively small number of people, but
there is a public free version that anybody can instantly start using
at \url{http://sagenb.org}.  Sage is software for doing serious number
crunching, so it is extremely resource intensive.  The notebook runs
on one computer the PI purchased in {\bf 2008} and has around 100,000
user accounts!  There are usually well over 50 people attempting to
use \url{http://sagenb.org} at any given time.

\begin{figure}[ht]
\begin{center}
\includegraphics[width=0.45\textwidth]{nb2}
\caption{A Screenshot Showing the Sage Notebook\label{fig:sagenb}}
\end{center}
\end{figure}

{\bf Goal of this project:} {\em Create a new web-based service called
  Salvus, which will be highly available, scalable, robust, and always
  fast.  Provide both free and non-free options, with revenue from the
  non-free option supporting hardware, maintenance, and core
  development of Sage. }

Instead of students, teachers, and researchers having to pay to buy
expensive commercial mathematical software and install it on their
computers, Salvus would provide the option to {\em reliably} and
efficiently use Sage over the web.  At many institutions, purchasing
computer software---especially mathematical software---is a
significant burden, and Sage has helped address this problem, but the
burden of installing, mainting, and upgrading Sage remains, and the
Sage notebook hasn't solved that problem because it is too slow and
unreliable.     This project has the potential to have a profound impact
on education at all levels. 

\section{Objectives}
%What is the project designed to accomplish?

The PI is actively working on the first technical stages of Salvus,
and plans to have a useful version up and running with similar
functionality to the existing Sage Notebook and a core subscriber base
by August 2013, with work during Summer 2013 partly funded by an NSF
grant.  This RRF would allow the PI to work fulltime straight through
until late December 2013 on the second stage of the Salvus project.

The specific goals would be to directly address some significant {\em
  functionality} that is completely missing from the current Sage
Notebook's design, and isn't a priority for the first stage of work on
Salvus.

\begin{enumerate}
\item {\bf Modern Interactive 2d and 3d Graphics}

\item {\bf User and Site-specific Customization}

\item {\bf Enterprise Customers and Data Security}

\end{enumerate}


\subsection{The Sage Notebook}
\begin{figure}
\begin{center}
\includegraphics[width=0.6\textwidth]{nb1}
\caption{Interactive Image Compression in the Sage Notebook\label{fig:interact}}
\end{center}
\end{figure}
The Sage notebook is an AJAX application, like Gmail or Google Maps.
It provides an interactive web-based worksheet in which one can enter
arbitrary Sage commands, see beautifully typeset output, create 2-D
and 3-D graphics, publish worksheets, and collaborate with other
users.

The PI and several UW undergraduates together developed the basic
implementation of the current version of the Sage notebook during an
extremely intense three-week coding session in Summer 2007.  This coding
work was motivated by UW's SIMUW program, which is an intense summer
mathematics program for about 25 high school students.  In 2007, the
PI taught a two-week SIMUW course using the Sage notebook on the Riemann
Hypothesis.  In 2008 he taught another 2-week SIMUW course on
Quantitative Finance using the notebook's new interactive controls
feature (see Figure~\ref{fig:interact}).

The Sage notebook is a much beloved ``killer application'' of Sage:
\begin{quote}
  In my opinion, SAGE's notebook is the real killer feature, which I
  don't recall to have seen in any other (commercial or not)
  software. I mean, this is the only scientific program that I've
  found, allowing such an easy collaborative job within local
  networks.

 -- Maurizio, the Sage mailing list.
\end{quote}

Professors at dozens of universities around the world are getting
excited about how they can leverage the Sage notebook in their
teaching.
\begin{quote}
With some colleagues in our University (Lyon, France) we have built a
project around Sage for undergraduate students... {\em And the University has
decided to support this project.} Good news.

Now we will be facing the problem to build a Sage configuration which
will work for say 200 students at the same time (students will use the
notebook), and prepare professors for Sage teaching. There are `some'
technical problems to solve...
 
-- T. Dumont, the Sage mailing list.
\end{quote}

The Sage notebook presently does not robustly scale to more than about
100 users at the same time, no matter how good the hosting hardware
is.  The PI proposes tovastly improving the robustness and scalability
of the notebook.  The PI is well positioned to succeed at this
project, since he is intimately familiar with all aspects of the
notebook codebase.


\section{Procedure}

% With what methods, materials, or tools will the objectives be met?
% If access to a particular location or institution is required for
% research or data collection, state whether permission has been
% obtained.


\section{Time Schedule}
% Provide a schedule showing how the proposed research can be
% accomplished during the desired support period. The support period
% is normally limited to one year. A no-cost extension of up to one
% year may be granted if requested and adequately justified.

The PI intends to work full time for three months (Sept 16--Dec 16, 2013) on this project.

\begin{enumerate}
\item Month 1: ??
\item Month 2: ??
\item Month 3: ??
\end{enumerate}


\section{Need for RRF Support}
% What other efforts have been made to find support for the project?
% How could the results of the work lead to further outside funding or
% commercial applications? How does this project address the mission
% of the Royalty Research Fund? For RRF Scholar applicants, provide
% documentation of teaching load (quarter, course number, title, and
% credits) in this section.

Though the PI has obtained funding for the Sage project from
Microsoft, Google, the National Science Foundation, prize money,
private donations, and contracts with industry for Sage development,
this has mostly been limited seed money.  The more Sage improves and
grows in quality to equal and exceed the commercial offerings, the
{\em easier} it becomes to obtain further outside funding and for Sage
to be useful in potentially thousands of educational and commercial
applications.

The goals of the current proposal are to greatly enhance Sage's level
of {\em robust availability} and provide a sustainable UW-based
revenue stream to support Sage development.  If successful, this will
have a huge and direct impact both on development of Sage and the
community of Sage users.  This project thus directly addresses the
mission of the Royalty Research Fund by providing a unique opportunity
to increase the PI's competitiveness for subsequent funding via
individual user and site-wide subscriptions to Salvus.

The last RRF that the PI received in 2009 made Sage attractive to the
{\em educational market}, which was a huge area of potential funding
that Sage had not had success in yet.  After that RRF, the Sage
project (and the PI) received a substantial CCLI Type 2 grant (see
\url{http://utmost.aimath.org/}) from the educational part of NSF.
Similarly, supporting the current proposal is likely to expand the
potential funding options for Sage dramatically, by very efficiently
providing a new key service that is in high demand, but nobody
currently makes available.



\end{document}
