\usepackage{amsmath}%
\usepackage{amsfonts}%
\usepackage{amssymb}%
\usepackage{amsthm}%

\hoffset=-.04\textwidth%
\textwidth=1.08\textwidth%
\voffset=-0.03\textheight
\textheight=1.06\textheight
%\voffset=0.5cm%
%\textheight=1.08\textheight%

\newcommand{\C}{\mathbf{C}}%
\newcommand{\F}{\mathbf{F}}%
\newcommand{\Q}{\mathbf{Q}}%
\newcommand{\Qbar}{\overline{\Q}}%
\newcommand{\Z}{\mathbf{Z}}%
\newcommand{\R}{\mathbf{R}}%
\newcommand{\T}{\mathbf{T}}%
\renewcommand{\H}{\mathrm{H}}%

\newcommand{\eps}{\varepsilon}%
\newcommand{\con}{\equiv}%
\newcommand{\isom}{\cong}%
\newcommand{\rhobar}{\overline{\rho}}
\newcommand{\tensor}{\otimes}


% ---- SHA ----
\DeclareFontEncoding{OT2}{}{} % to enable usage of cyrillic fonts
  \newcommand{\textcyr}[1]{%
    {\fontencoding{OT2}\fontfamily{wncyr}\fontseries{m}\fontshape{n}%
     \selectfont #1}}%
\newcommand{\Sha}{{\mbox{\textcyr{Sh}}}}%
\newcommand{\tor}{\mbox{\scriptsize\rm tor}}

\newcommand{\myname}{William Stein}
\newcommand{\phone}{{\sf (206) 419-0925}}
\newcommand{\email}{{\sf wstein@gmail.com}}
\newcommand{\www}{{\sf http://wstein.org}}
\newcommand{\address}{}
\usepackage{fancyhdr,ifthen}
\pagestyle{fancy}
\cfoot{\thepage}  % no footers (in pagestyle fancy)
% running left heading
\lhead{\bfseries\LARGE\em \noindent{}\hspace{-.2em}\myname{}
        \hfill \thisdocument\vspace{-.2ex}\\}
% running right heading
%\newcommand{\spc}{1.31em}
\newcommand{\spc}{1em}

%\rhead{\em {\small{\phone{}}} \hfill $\cdot$\hfill
% \email{} \hfill $\cdot$\hfill \www{}}

\rhead{\em {\small{\phone{}}} \hfill \email{} \hfill \www{}}

\setlength{\headheight}{5ex}
\newcommand{\mainhead}[1]{\begin{center}{\Large \bf #1}\end{center}}
\newcommand{\head}[1]{\vspace{1.5ex}\par\noindent{\large \bf #1}\par\noindent}
\newcommand{\subhead}[1]{\vspace{2ex}\par\noindent{\sl #1}\vspace{1ex}\par\noindent{}}
\newcommand{\ptitle}{\sl}

\newcommand{\hra}{\hookrightarrow}


%%%% Theoremstyles
\theoremstyle{plain}
\newtheorem{theorem}{Theorem}[section]
\newtheorem{proposition}[theorem]{Proposition}
\newtheorem{corollary}[theorem]{Corollary}
\newtheorem{claim}[theorem]{Claim}
\newtheorem{lemma}[theorem]{Lemma}
\newtheorem{conjecture}[theorem]{Conjecture}

\theoremstyle{definition}
\newtheorem{definition}[theorem]{Definition}
\newtheorem{algorithm}[theorem]{Algorithm}
\newtheorem{question}[theorem]{Question}
\newtheorem{problem}[theorem]{Problem}
\newtheorem{goal}[theorem]{Goal}

\theoremstyle{remark}
\newtheorem{remark}[theorem]{Remark}
\newtheorem{remarks}[theorem]{Remarks}
\newtheorem{example}[theorem]{Example}
\newtheorem{exercise}[theorem]{Exercise}

\DeclareMathOperator{\End}{End}%
\DeclareMathOperator{\Tr}{Tr}%
\DeclareMathOperator{\Res}{Res}%
\DeclareMathOperator{\res}{res}%
\DeclareMathOperator{\BSD}{BSD}%
\DeclareMathOperator{\Gal}{Gal}%
\DeclareMathOperator{\GL}{GL}%
\DeclareMathOperator{\Aut}{Aut}%
\DeclareMathOperator{\Reg}{Reg}%
\DeclareMathOperator{\Vis}{Vis}%
\DeclareMathOperator{\Ker}{Ker}%
\DeclareMathOperator{\Coker}{Coker}%
\DeclareMathOperator{\Sel}{Sel}%
\DeclareMathOperator{\ord}{ord}%
\DeclareMathOperator{\new}{new}%
\DeclareMathOperator{\an}{an}%

